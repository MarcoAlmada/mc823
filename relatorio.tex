\documentclass[11pt, brazil]{article} % use larger type; default would be 10pt

\usepackage[brazilian]{babel}
\usepackage{ae,aecompl}
\usepackage[T1]{fontenc}
\usepackage[utf8]{inputenc} % set input encoding (not needed with XeLaTeX)
\usepackage{graphicx}
\setcounter{secnumdepth}{3}
\setcounter{tocdepth}{3}
\usepackage{array}
\usepackage{amsmath}
\usepackage{amssymb}
\usepackage{amsthm}
\PassOptionsToPackage{normalem}{ulem}
\usepackage{ulem}

%\usepackage[round]{natbib}

\makeatletter
\providecommand{\tabularnewline}{\\}

%
% Required packages

\usepackage{epsfig}
\usepackage{colortbl}\usepackage{paralist}\usepackage{multirow}\@ifundefined{definecolor}
 {\@ifundefined{definecolor}
 {\@ifundefined{definecolor}
 {\@ifundefined{definecolor}
 {\usepackage{color}}{}
}{}
}{}
}{}
\usepackage{indentfirst}% retira padrao americano de paragrafos
\usepackage{multicol}\usepackage[linkbordercolor={1 1 1},urlcolor=black,colorlinks=false]{hyperref}% links
\usepackage{amsfonts}%matematicos
\usepackage{enumerate}%enumeração
\usepackage{wrapfig}% blocos figuras
\usepackage{subfig}
\usepackage{indentfirst}
\usepackage{ae}
\usepackage{aecompl}
\usepackage{pslatex}
\usepackage{pstricks}
\usepackage{epsf}
\usepackage{pstricks-add}
\usepackage{fullpage}

\makeatother

%%% PAGE DIMENSIONS
%\usepackage{geometry} % to change the page dimensions
%\geometry{a4paper} % or letterpaper (US) or a5paper or....

%%% HEADERS & FOOTERS
\usepackage{fancyhdr} % This should be set AFTER setting up the page geometry
\pagestyle{fancy} % options: empty , plain , fancy
\renewcommand{\headrulewidth}{0pt} % customise the layout...
\lhead{}\chead{}\rhead{}
\lfoot{}\cfoot{\thepage}\rfoot{}

%%% SECTION TITLE APPEARANCE
\usepackage{sectsty}
\allsectionsfont{\sffamily\mdseries\upshape} % (See the fntguide.pdf for font help)
% (This matches ConTeXt defaults)

%%% The "real" document content comes below...

\title{MC823 -- 1º Relatório}
%\subtitle{Segundo Semestre de 2011}
\author{Marco Antônio Lasmar Almada -- RA 092208 \\ Patricia Fernanda Hongo -- RA 103715}
\date{} % Activate to display a given date or no date (if empty),
         % otherwise the current date is printed 

\begin{document}

\maketitle

\section{Introdução}

O presente relatório tem como objetivo descrever as atividades desempenhadas pela dupla durante a implementação do primeiro projeto de MC823 no primeirro semestre de 2013. Tal projeto consistia na implementação de um sistema para consulta e controle de estoque em uma biblioteca, utilizando uma arquitetura cliente-servidor. No primeiro projeto, a comunicação entre servidor e clientes será realizada através do protocolo TCP, com uma implementação de servidor concorrente.

O servidor armazena as seguintes informações sobre livros:
\begin{itemize}
  \item ISBN;
  \item Título;
  \item Descrição;
  \item Estoque;
  \item Autor;
  \item Editora;
  \item Ano;  
\end{itemize}

O cliente pode efetuar três tipos de operações: listar todos os livros (o cliente pode escolher entre receber todos os dados de cada livro ou somente ISBN e título), obter dados de um livro cujo ISBN é conhecido (pode pedir a descrição, o número de livros em estoque ou todos os dados do livro) ou alterar o número de unidades em estoque de um livro cujo ISBN seja conhecido.

O mesmo sistema será posteriormente reimplementado utilizando o protocolo UDP, com a finalidade de identificar as semelhanças e as diferenças no funcionamento das duas implementações. Assim, os dados obtidos durante o primeiro projeto servirão de referência para o projeto seguinte.

\section{Implementação}

Tanto o servidor quanto o cliente foram implementados em linguagem C, e os códigos serão apresentados como anexos deste relatório.
O banco de dados foi implementado como um arquivo \texttt{.txt}, que passa por uma etapa de pré-processamento no servidor: o arquivo com os dados dos livros é lido pelo servidor, que armazena os dados de cada livro em um vetor de \texttt{struct livro}s, para que não seja necessária uma nova consulta ao banco de dados a cada operação. 
O pré-processamento pareceu uma opção viável, uma vez que não seria necessário lidar com um número grande de livros. 
Além disso, o servidor fica na escuta em um \emph{socket} TCP, aguardando algum cliente se conectar. 
Ao receber uma mensagem de um cliente, o servidor faz um \texttt{fork} para processá-la, recebendo o código da operação selecionada pelo cliente e os eventuais parâmetros.
O vetor de livros é então varrido em busca dos dados desejados, que são adicionados a um \emph{buffer} de mensagens.
Ao final do processamento da operação, o \emph{buffer} é enfiado ao cliente responsável pela requisição, por TCP.

O cliente, ao ser inicializado, fornece ao usuário um menu com as funcionalidades disponíveis ao programa, elencadas na Introdução e na especificação do projeto. 
O usuário seleciona uma opção e, se for o caso, fornece os parâmetros necessários.
O ISBN é o tipo de parâmetro mais comum das operações que o pedem, mas a alteração de estoque também pede uma senha de acesso e o novo valor do estoque. 
Com base na opção selecionada e em eventuais parâmetros, o cliente constrói uma mensagem no \emph{buffer} específico. 
A solicitação é então enviada para o servidor por TCP, e a mensagem de volta é processada e impressa em tela.

%\section{Discussão da Solução}

\section{Resultados}

\section{Conclusão}

Os \emph{sockets} TCP funcionam bem para a comunicação entre processos. Embora os testes tenham sido realizados apenas em máquinas em uma mesma rede local, é possível supor que se manterá a funcionalidade quando as conexões forem realizadas entre máquinas em redes diferentes, graças às garantias oferecidas pelo protocolo TCP. Assim, a versão do sistema empregando o protocolo UDP provavelmente exigirá a implementação de salvaguardas adicionais para garantir o mesmo nível de confiabilidade atualmente obtido somente com o uso de instruções básicas para a criação dos \emph{sockets} TCP.

A implementação realizada pode ter alguns problemas com consistência dos dados e robustez. O uso de um sistema gerenciador de banco de dados, assim como o emprego de diretivas de semáforo ou \emph{mutexes}, eliminaria a chance de dois processos-filhos do servidor possuírem valores diferentes para o estoque, assim como facilitaria a criação de uma operação de adição de livros ao sistema. O problema também poderia ser resolvido utilizando \emph{memory maps}, mas a pequena escala planejada para o projeto, assim como o intuito pedagógico, contribuiram para que tal solução fosse vista como algo de baixa prioridade. Também não foram adotadas medidas de segurança para o acesso com senha à base de dados.

\section{Referências}

HALL, Brian. \emph{Beej's Guide to Network Programming} -- Using Internet Sockets. Versão 3.0.15, 3 de julho de 2012. Disponível em <http://beej.us/guide/bgnet/>.

\section*{Anexo 1: Código do Servidor}

\begin{verbatim}
#include <stdio.h>
#include <stdlib.h>
#include <unistd.h>
#include <errno.h>
#include <string.h>
#include <sys/types.h>
#include <sys/socket.h>
#include <netinet/in.h>
#include <netdb.h>
#include <arpa/inet.h>
#include <sys/wait.h>
#include <signal.h>
#include <sys/time.h>

#define PORT "3490"  // the port users will be connecting to

#define BACKLOG 10	 // how many pending connections queue will hold

#define MAXDATASIZE 200123 // max number of bytes we can get at once

#define MAXBIBSIZE 10

struct livro{
  char ISBN[14];
  char titulo[1000];
  char descricao[100123];
  int estoque;
  char autor[100];
  char editora[100];
  int ano;
};

void sigchld_handler(int s)
{
  while(waitpid(-1, NULL, WNOHANG) > 0);
}

// get sockaddr, IPv4 or IPv6:
void *get_in_addr(struct sockaddr *sa)
{
  if (sa->sa_family == AF_INET) {
    return &(((struct sockaddr_in*)sa)->sin_addr);
  }

  return &(((struct sockaddr_in6*)sa)->sin6_addr);
}

int main(void)
{
  int sockfd, new_fd;  // listen on sock_fd, new connection on new_fd
  struct addrinfo hints, *servinfo, *p;
  struct sockaddr_storage their_addr; // connector's address information
  socklen_t sin_size;
  struct sigaction sa;
  int yes=1;
  char s[INET6_ADDRSTRLEN];
  int rv;

  memset(&hints, 0, sizeof hints);
  hints.ai_family = AF_UNSPEC;
  hints.ai_socktype = SOCK_STREAM;
  hints.ai_flags = AI_PASSIVE; // use my IP

  if ((rv = getaddrinfo(NULL, PORT, &hints, &servinfo)) != 0) {
    fprintf(stderr, "getaddrinfo: %s\n", gai_strerror(rv));
    return 1;
  }

  // loop through all the results and bind to the first we can
  for(p = servinfo; p != NULL; p = p->ai_next) {
    if ((sockfd = socket(p->ai_family, p->ai_socktype,p->ai_protocol)) == -1) {
      perror("server: socket");
      continue;
    }

    if (setsockopt(sockfd, SOL_SOCKET, SO_REUSEADDR, &yes, sizeof(int)) == -1) {
      perror("setsockopt");
      exit(1);
    }

    if (bind(sockfd, p->ai_addr, p->ai_addrlen) == -1) {
      close(sockfd);
      perror("server: bind");
      continue;
    }

    break;
  }

  if (p == NULL)  {
    fprintf(stderr, "server: failed to bind\n");
    return 2;
  }

  freeaddrinfo(servinfo); // all done with this structure

  if (listen(sockfd, BACKLOG) == -1) {
    perror("listen");
    exit(1);
  }

  sa.sa_handler = sigchld_handler; // reap all dead processes
  sigemptyset(&sa.sa_mask);
  sa.sa_flags = SA_RESTART;
  if (sigaction(SIGCHLD, &sa, NULL) == -1) {
    perror("sigaction");
    exit(1);
  }

  printf("server: waiting for connections...\n");

  //preprocessamento
  struct livro biblioteca[MAXBIBSIZE];
  char aux[MAXDATASIZE];
  FILE *dados = fopen("dados.txt", "r");
  int i = 0, j;
  while(!feof(dados) ){
    fgets(biblioteca[i].ISBN, 11, dados);
    fgetc(dados);
		
    fgets(biblioteca[i].titulo, 1000, dados);
    fgets(biblioteca[i].descricao, MAXDATASIZE-1, dados);
    fgets(aux, MAXDATASIZE, dados);
    biblioteca[i].estoque = atoi(aux);
    fgets(biblioteca[i].autor, MAXDATASIZE-1, dados);
    fgets(biblioteca[i].editora, MAXDATASIZE-1, dados);
    fgets(aux, MAXDATASIZE-1, dados);
    biblioteca[i].ano = atoi(aux);
    i++;
		
  }
  int total_livros = i;
  printf("--> %d\n", total_livros);
	
  //memory map
  //char *addr;
  //addr = mmap(NULL, sizeof(biblioteca), PROT_WRITE | PROT_READ, MAP_SHARED,
  //      int fd, off_t offset);
	
  int bytes_rcv;
  char buf[MAXDATASIZE];
  int opt, cont, qte;
  char ISBN[20];
  FILE *db = fopen("dados.txt", "rw");
  while(1) {  // main accept() loop
    sin_size = sizeof their_addr;
    new_fd = accept(sockfd, (struct sockaddr *)&their_addr, &sin_size);
    if (new_fd == -1) {
      perror("accept");
      continue;
    }

  inet_ntop(their_addr.ss_family,get_in_addr((struct sockaddr *)&their_addr),s, sizeof s);
  printf("server: got connection from %s\n", s);
  if (!fork()) { // this is the child process
    close(sockfd); // child doesn't need the listener
    //LOOP QUE PROCESSA REQUISICOES
    char bufs[MAXDATASIZE];

    /**/while(1){		
      if ((bytes_rcv = recv(new_fd, buf, MAXDATASIZE-1, 0)) == -1) {
        perror("erro no recv");
	break;
      }
		    	
      int mudou = 0; //mudou estoque
	    	
      //pega tempo
      struct timeval tempo_in, tempo_fim;
      gettimeofday(&tempo_in, NULL);
		    	
      sscanf(buf, "%d", &opt);
		    	
      if(opt == 1){
	bufs[0] = '\0';
	strcat(bufs, "0 ");
	for(i = 0; i < total_livros; ++i){
  	  strcat(bufs, biblioteca[i].ISBN);
	  strcat(bufs, " ");
	  strcat(bufs, biblioteca[i].titulo);
	  }
      } 
		    	
      else if(opt == 2){
 	bufs[0] = '\0';
	strcat(bufs, "0 ");
	sscanf(buf, "%d %s", &opt, ISBN);
	for(i = 0; i < total_livros; ++i){
	  if(strcmp(biblioteca[i].ISBN, ISBN) == 0){
            strcat(bufs, biblioteca[i].descricao);
	    break;
	  }
        }
      }
		    	
      else if(opt == 3){
        bufs[0] = '\0';
	strcat(bufs, "0 ");
	sscanf(buf, "%d %s", &opt, ISBN);
        for(i = 0; i < total_livros; ++i){
	  if(strcmp(biblioteca[i].ISBN, ISBN) == 0){
	    strcat(bufs, biblioteca[i].ISBN);
	    strcat(bufs, "\n");
  	    strcat(bufs, biblioteca[i].titulo);
            strcat(bufs, biblioteca[i].descricao);
            sprintf(aux, "%d\n", biblioteca[i].estoque);
  	    strcat(bufs, aux);
            strcat(bufs, biblioteca[i].autor);
    	    strcat(bufs, biblioteca[i].editora);
	    sprintf(aux, "%d\n", biblioteca[i].ano);
	    strcat(bufs, aux);
	    break;
	  }		    				
        }
      }
		    
      else if(opt == 4){
        bufs[0] = '\0';
 	strcat(bufs, "0 ");
	for(i = 0; i < total_livros; ++i){
	  strcat(bufs, biblioteca[i].ISBN);
 	  strcat(bufs, "\n");
	  strcat(bufs, biblioteca[i].titulo);
          strcat(bufs, biblioteca[i].descricao);
	  sprintf(aux, "%d\n", biblioteca[i].estoque);
    	  strcat(bufs, aux);
    	  strcat(bufs, biblioteca[i].autor);
	  strcat(bufs, biblioteca[i].editora);
	  sprintf(aux, "%d\n", biblioteca[i].ano);
          strcat(bufs, aux);
        }	
      }
		    	
      else if(opt == 6){
        bufs[0] = '\0';
        strcat(bufs, "0 ");
        sscanf(buf, "%d %s", &opt, ISBN);
    	for(i = 0; i < total_livros; ++i){
      	  if(strcmp(biblioteca[i].ISBN, ISBN) == 0){
	    sprintf(aux, "%d\n", biblioteca[i].estoque);
    	    strcat(bufs, aux);
      	    break;
	  }		    				
        }
      }
		    	
      else if(opt == 5){
        bufs[0] = '\0';
  	strcat(bufs, "0 ");
        sscanf(buf, "%d %s %d", &opt, ISBN, &qte);
        mudou = 1;
        for(i = 0; i < total_livros; ++i){
	  if(strcmp(biblioteca[i].ISBN, ISBN) == 0){
	    biblioteca[i].estoque = qte;
	    break;
          }		    				
    	}
      }
                        		    	
      //pega tempo final
      gettimeofday(&tempo_fim, NULL);
      double tempo1, tempo2;
      tempo1 = tempo_in.tv_sec + 0.000001*tempo_in.tv_usec;
      tempo2 = tempo_fim.tv_sec + 0.000001*tempo_fim.tv_usec;
      printf("\nTempo total: %lf %d\n", tempo2-tempo1, (int) getpid() );
				
      if (send(new_fd, bufs, strlen(bufs)+1, 0) == -1)
	perror("send");
					
      if(mudou){
	bufs[0] = '\0';
   	for(i = 0; i < total_livros-1; ++i){
   	  strcat(bufs, biblioteca[i].ISBN);
   	  strcat(bufs, "\n");
   	  strcat(bufs, biblioteca[i].titulo);
   	  strcat(bufs, biblioteca[i].descricao);
   	  sprintf(aux, "%d\n", biblioteca[i].estoque);
   	  strcat(bufs, aux);
   	  strcat(bufs, biblioteca[i].autor);
   	  strcat(bufs, biblioteca[i].editora);
   	  sprintf(aux, "%d\n", biblioteca[i].ano);
   	  strcat(bufs, aux);			
    	}
  	FILE *fout = fopen("dados.txt", "w");
   	fprintf(fout, "%s", bufs);
    	fclose(fout);
     }
     /**/}
						
     close(new_fd);
     exit(0);
     }
     close(new_fd);  // parent doesn't need this
     }

     return 0;
}
\end{verbatim}

\section*{Anexo 2: Código do Cliente}

\begin{verbatim}
#include <stdio.h>
#include <stdlib.h>
#include <unistd.h>
#include <errno.h>
#include <string.h>
#include <netdb.h>
#include <sys/types.h>
#include <netinet/in.h>
#include <sys/socket.h>
#include <sys/time.h>
#include <arpa/inet.h>

#define PORT "3490" // the port client will be connecting to 

#define MAXDATASIZE 200123 // max number of bytes we can get at once 

// get sockaddr, IPv4 or IPv6:
void *get_in_addr(struct sockaddr *sa)
{
    if (sa->sa_family == AF_INET) {
        return &(((struct sockaddr_in*)sa)->sin_addr);
    }

    return &(((struct sockaddr_in6*)sa)->sin6_addr);
}

int main(int argc, char *argv[])
{
    int sockfd, numbytes;  
    char buf[MAXDATASIZE];
    struct addrinfo hints, *servinfo, *p;
    int rv;
    char s[INET6_ADDRSTRLEN];
    int estoque;

    if (argc != 2) {
        fprintf(stderr,"usage: client hostname\n");
        exit(1);
    }

    memset(&hints, 0, sizeof hints);
    hints.ai_family = AF_UNSPEC;
    hints.ai_socktype = SOCK_STREAM;

    if ((rv = getaddrinfo(argv[1], PORT, &hints, &servinfo)) != 0) {
        fprintf(stderr, "getaddrinfo: %s\n", gai_strerror(rv));
        return 1;
    }

    // loop through all the results and connect to the first we can
    for(p = servinfo; p != NULL; p = p->ai_next) {
        if ((sockfd = socket(p->ai_family, p->ai_socktype,
                p->ai_protocol)) == -1) {
            perror("client: socket");
            continue;
        }

        if (connect(sockfd, p->ai_addr, p->ai_addrlen) == -1) {
            close(sockfd);
            perror("client: connect");
            continue;
        }

        break;
    }

    if (p == NULL) {
        fprintf(stderr, "client: failed to connect\n");
        return 2;
    }

    inet_ntop(p->ai_family, get_in_addr((struct sockaddr *)p->ai_addr),
            s, sizeof s);
    printf("client: connecting to %s\n", s);

    freeaddrinfo(servinfo); // all done with this structure
	
	int opt, cont;
	char ISBN[20];
	char msg[MAXDATASIZE];
	char pass[20];
	int bytes_sent, len, bytes_rcv;
	
	while(1){
		ISBN[0] = '\0';
		//pseudo user interface
		printf("Escolha uma opcao:\n");
		printf("1- listar todos os ISBN e seus respectivos titulos\n");
		printf("2- dado o ISBN de um livro, retornar descricao\n");
		printf("3- dado o ISBN de um livro, retornar todas as informacoes do livro\n");
		printf("4- listar todas as informacoes de todos os livros\n");
		printf("5- alterar numero de exemplares em estoque\n");
		printf("6- dado o ISBN de um livro, retornar numero de exemplares em estoque\n");
		printf("7- fechar cliente\n");
		scanf("%d", &opt);
		if(opt == 7) break;
		if(opt == 5){
			printf("Digite a senha\n");
			scanf(" %s", pass);
			if(strcmp(pass, "senhalivraria") != 0)
			continue;
			else{
			  printf("Digite o ISBN\n");
			  scanf(" %s", ISBN);
			  printf("Digite o novo valor do estoque\n");
			  scanf("%d", &estoque);
			}
		}
		if(opt == 2 || opt == 3 || opt == 6){
			printf("Digite o ISBN\n");
			scanf(" %s", ISBN);
		}
		
		//escreve requisicao
		if (opt == 5) sprintf(msg, "%d %s %d", opt, ISBN, estoque);
		else sprintf(msg, "%d %s", opt, ISBN);  
		
		
		//pega tempo inicial
		struct timeval tempo_in, tempo_fim;
		gettimeofday(&tempo_in, NULL);
		
		len = strlen(msg);
		if((bytes_sent = send(sockfd, msg, len, 0)) != len){
			printf("erro no send\n");
			continue;	
		}
	
		//le resultado
		while(1){
		    if ((bytes_rcv = recv(sockfd, buf, MAXDATASIZE-1, 0)) == -1) {
		        perror("erro no recv");
		        break;
		    }
		    sscanf(buf, "%d", &cont);    		    
		    printf("%s", buf);
		    if(cont == 0) break;
		    //		    printf("%d\n", cont);
		    //       		    printf("%s", buf+2);
		    //		    printf("\n%d", cont);
		}
		
		//pega tempo final
		gettimeofday(&tempo_fim, NULL);
		double tempo1, tempo2;
		tempo1 = tempo_in.tv_sec + 0.000001*tempo_in.tv_usec;
		tempo2 = tempo_fim.tv_sec + 0.000001*tempo_fim.tv_usec;
		printf("\nTempo total: %lf\n", tempo2-tempo1);
	}

    close(sockfd);

    return 0;
}
\end{verbatim}

\end{document}