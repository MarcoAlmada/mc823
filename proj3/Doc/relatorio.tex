\documentclass[11pt, brazil]{article} % use larger type; default would be 10pt

\usepackage[brazilian]{babel}
\usepackage{ae,aecompl}
\usepackage[T1]{fontenc}
\usepackage[utf8]{inputenc} % set input encoding (not needed with XeLaTeX)
\usepackage{graphicx}
\setcounter{secnumdepth}{3}
\setcounter{tocdepth}{3}
\usepackage{array}
\usepackage{amsmath}
\usepackage{amssymb}
\usepackage{amsthm}
\PassOptionsToPackage{normalem}{ulem}
\usepackage{ulem}

%\usepackage[round]{natbib}

\makeatletter
\providecommand{\tabularnewline}{\\}

%
% Required packages

\usepackage{epsfig}
\usepackage{colortbl}\usepackage{paralist}\usepackage{multirow}\@ifundefined{definecolor}
 {\@ifundefined{definecolor}
 {\@ifundefined{definecolor}
 {\@ifundefined{definecolor}
 {\usepackage{color}}{}
}{}
}{}
}{}
\usepackage{indentfirst}% retira padrao americano de paragrafos
\usepackage{multicol}\usepackage[linkbordercolor={1 1 1},urlcolor=black,colorlinks=false]{hyperref}% links
\usepackage{amsfonts}%matematicos
\usepackage{enumerate}%enumeração
\usepackage{wrapfig}% blocos figuras
\usepackage{subfig}
\usepackage{indentfirst}
\usepackage{ae}
\usepackage{aecompl}
\usepackage{pslatex}
\usepackage{pstricks}
\usepackage{epsf}
\usepackage{pstricks-add}
\usepackage{fullpage}

\makeatother

%%% PAGE DIMENSIONS
%\usepackage{geometry} % to change the page dimensions
%\geometry{a4paper} % or letterpaper (US) or a5paper or....

%%% HEADERS & FOOTERS
\usepackage{fancyhdr} % This should be set AFTER setting up the page geometry
\pagestyle{fancy} % options: empty , plain , fancy
\renewcommand{\headrulewidth}{0pt} % customise the layout...
\lhead{}\chead{}\rhead{}
\lfoot{}\cfoot{\thepage}\rfoot{}

%%% SECTION TITLE APPEARANCE
\usepackage{sectsty}
\allsectionsfont{\sffamily\mdseries\upshape} % (See the fntguide.pdf for font help)
% (This matches ConTeXt defaults)

%%% The "real" document content comes below...

\title{MC823 -- 1º Relatório}
%\subtitle{Segundo Semestre de 2011}
\author{Marco Antônio Lasmar Almada -- RA 092208 \\ Patricia Fernanda Hongo -- RA 103715}
\date{} % Activate to display a given date or no date (if empty),
         % otherwise the current date is printed 

\begin{document}

\maketitle

\section{Introdução}

O presente relatório tem como objetivo descrever a implementação e os resultados do terceiro projeto de MC823 no primeiro semestre de 2013, comparando-os com o resultados obtidos nos dois primeiros projetos. Tal como os Projetos 1 e 2, este projeto consistia na implementação de um sistema para consulta e controle de estoque em uma biblioteca, utilizando uma arquitetura cliente-servidor. Entretanto, a implementação o sistema desta vez utilizou a invocação remota de métodos da linguagem Java em vez de lidar diretamente com os \emph{sockets}.

O servidor armazena as seguintes informações sobre livros:
\begin{itemize}
  \item ISBN;
  \item Título;
  \item Descrição;
  \item Estoque;
  \item Autor;
  \item Editora;
  \item Ano;  
\end{itemize}

O cliente pode efetuar três tipos de operações: listar todos os livros (o cliente pode escolher entre receber todos os dados de cada livro ou somente ISBN e título), obter dados de um livro cujo ISBN é conhecido (pode pedir a descrição, o número de livros em estoque ou todos os dados do livro) ou alterar o número de unidades em estoque de um livro cujo ISBN seja conhecido.

\section{Implementação}


A implementação do sistema de livraria neste projeto foi bastate semelhante à implementação dos projetos anteriores, porém dessa vez aproveitando o nível de abstração mais alto da linguagem Java. Os códigos do cliente, do servidor e da interface para as diferentes operações com a biblioteca.

A interface remota \texttt{Biblio} define os métodos que poderão ser invocados remotamente pelo cliente e as exceções remotas que estes retornam caso necessário. Suas implementações são realizadas no código do servidor, e o cliente faz as chamadas de métodos.

O pré-processamento utilizado na base de dados da livraria foi bem semelhante ao das implementações em C: o arquivo \texttt{.txt} com os dados dos livros é lido pelo servidor, que armazena os dados de cada livro em um \texttt{arrayList} para cada categoria das elencadas anteriormente.
Dessa forma, não é necessária uma nova consulta ao banco de dados a cada operação requisitada pelo cliente.

Após o pré-processamento, o servidor fica no aguardo das chamadas remotas, escutando um \emph{stub} criado a partir da declaração da interface remota. Quando o cliente faz chamadas de métodos, o servidor executa suas implementações dos métodos da interface, enviando resposta de tipo adequado. Ao final da execução de um método, o servidor imprime o tempo decorrido entre o recebimento por ele da chamada e o fim da execução do método.

O cliente, ao ser inicializado, fornece ao usuário um menu com as funcionalidades disponíveis
ao programa, elencadas na Introdução e na especificação do projeto. 
O usuário seleciona uma opção e fornece os parâmetros requeridos pela operação.
Com base na opção selecionada e em eventuais parâmetros, o cliente faz uma chamada a um método remoto, por meio de um \emph{stub} pelo qual ele pode chamar os métodos da interface. Ao receber a resposta, o cliente imprime o tempo decorrido entre o envio da solicitação e o recebimento do retorno.

Para que a comunicação entre cliente e servidor funcione, é necessário executar o registro de objetos remotos do RMI, a partir do comando \texttt{rmiregistry}. É graças a ele que o cliente consegue obter a referência aos objetos remotos.

\section{Vantagens e Desvantagens da Solução}

O pré-processamento pareceu uma opção viável, uma vez que não seria necessário lidar com um
número grande de livros. Assim, o consumo de memória é relativamente pequeno e há um ganho em desempenho. 
Ao final de sua execução, o servidor salva seus dados em disco, efetivamente atualizando a versão permanente do banco de dados.
Como a chance de panes no servidor é relativamente baixa, os riscos associados a essa escolha são poucos, mas um sistema mais robusto salvaria os dados de uma maneira mais eficiente. 
O uso de um sistema gerenciador de banco de dados facilitaria o tratamento desse risco.

O emprego de constructos com maior nível de abstração simplificou bastante a implementação do código. Porém, isso se refletiu em um \emph{overhead} maior, tanto no código do cliente e do servidor em comparação com suas versões em C quanto na comunicação em RMI em comparação com os \emph{sockets}.

Não foram adotadas medidas de segurança para o acesso com senha à base de dados. Também no aspecto da segurança, decidiu-se não usar um \texttt{SecurityManager} no código do servidor, que serviria para a proteção dos recursos de sistema. 
Além de apresentar um risco de segurança, isso também resulta em uma restrição das funcionalidades disponíveis, devido às medidas de segurança adotadas pelo RMI.

\section{Resultados}

Os resultados estão presentes na tabelas anexas a este documento. Para todas as 6 operações,
as medições foram obtidas a partir da realização de 50 requisições sequenciais originadas
por um único cliente.

O tempo de consulta é o tempo decorrido entre imediatamente após o servidor receber a requisição e imediatamente antes de ele enviar a mensagem com a resposta. O tempo total é medido desde antes do envio da mensagem pelo cliente até depois do recebimento da resposta por ele, e o tempo de comunicação é metade da diferença entre esses tempos.

Como nos projetos 1 e 2, o tempo de comunicação representou a maior parte do tempo total de atividade do cliente. Para a maioria das operações, o tempo médio de comunicação foi em torno de uma ordem de grandeza maior do que os verificados com a manipulação direta de \emph{sockets}, exceto para a operação 5, em que o tipo de retorno definido pela interface é \texttt{void}; nesse caso, a implementação em RMI apresentou vantagem.

O tempo de consulta da implementação em Java foi superior em todos os casos, graças às diferenças entre o nível de abstração em relação ao código em C. Porém, a diferença não foi o bastante para compensar o impacto no tempo total da operação 5 da comunicação significativamente mais ágil proporcionada pelo RMI nesse caso.

\section{Conclusão}

Uma vez que os testes foram realizados na mesma rede e aproximadamente no mesmo horário, podemos concluir que a maior parte da diferença nos tempos de comunicação se deve às diferenças entre os protocolos, já que os niveis de uso da rede são aproximadamente os mesmos.

A diferença entre os tempos de comunicação com o UDP e com o TCP fica mais pronunciada com o aumento do tamanho da mensagem de retorno. Ainda assim, quando há efetivamente algo a se retornar, a comunicação por meio de \emph{sockets} é significativamente mais rápida do que o uso de RMI.

Desta forma, o experimento permitiu ao grupo a visualização clara do \emph{trade-off} entre abstração e desempenho: adicionar mais camadas de abstração facilita o trabalho de desenvolvimento e implementação dos programas, mas com um impacto na velocidade de funcionamento e conexão dos programas. Usar a invocação de métodos remotos simplifica a implementação de código para a comunicação entre clientes e servidores, mas resulta em desempenho menor que aquele que pode ser obtido ao lidar com código de mais baixo nível.


\section{Referências}

SUN. Trail: RMI. \emph{The Java Tutorials}. Disponível em <http://docs.oracle.com/javase/tutorial/rmi/index.html>.

\section*{Anexo 1: Código do Servidor}

\begin{verbatim}
package example.biblio;
	
import java.rmi.registry.Registry;
import java.rmi.registry.LocateRegistry;
import java.rmi.RemoteException;
import java.rmi.server.UnicastRemoteObject;

import java.util.ArrayList;
import java.util.Arrays;

import java.io.File;
import java.io.FileNotFoundException;
import java.util.Scanner;
	
public class Server implements Biblio {
	
    public Server() {
    	preparaDados();	
    }
    
    //ArrayLists armazenarao banco de dados da biblioteca
    private ArrayList<String> ISBN = new ArrayList<String> ();
    private ArrayList<String> titulo = new ArrayList<String> ();
    private ArrayList<String> descricao = new ArrayList<String> ();
    private ArrayList<Integer> estoque = new ArrayList<Integer> ();
    private ArrayList<String> autor = new ArrayList<String> ();
    private ArrayList<String> editora = new ArrayList<String> ();
    private ArrayList<Integer> ano = new ArrayList<Integer> ();
    
    //variaveis para contagem de tempo
    long inicio = 0, fim = 0;
    double sec;
    
    /*Pre-processamento: Os dados do banco serao armazenados na
      nas 7 ArrayList declaradas acima*/
    private void preparaDados(){
        File file = new File("dados.txt");
        try{
            Scanner sc = new Scanner(file);
            while (sc.hasNextLine()) {
                ISBN.add(sc.nextLine());
                titulo.add(sc.nextLine());
                descricao.add(sc.nextLine());
                estoque.add(Integer.parseInt(sc.nextLine()));
                autor.add(sc.nextLine());
                editora.add(sc.nextLine());            
                ano.add(Integer.parseInt(sc.nextLine()));                
            }
            sc.close();
        } catch (FileNotFoundException e) {
            e.printStackTrace();
        }
    }
	
    //Operacao 1
    public String listaISBN() {
    	//obtem tempo inicial
    	inicio = System.nanoTime();
    	
    	//processa requisicao
    	String response = Arrays.toString(ISBN.toArray());
    	
    	//obtem tempo final e imprime
    	fim = System.nanoTime();
        sec = fim-inicio;
        sec /= 1000000000;
        System.out.print("Tempo da operacao: ");
        System.out.printf("%.9f\n", sec);

        return response;
    }
    
    //Operacao 2
    public String retornaDescricao(String isbn) {
    	//obtem tempo inicial
    	inicio = System.nanoTime();
    		
    	//processa requisicao
    	String response = new String();
        for (int i = 0; i < ISBN.size(); i++) {
            String str = ISBN.get(i);
            if(str.compareTo(isbn) == 0){
                response = descricao.get(i);
            }
        }
	
        //obtem tempo final e imprime
    	fim = System.nanoTime();
        sec = fim-inicio;
        sec /= 1000000000;
        System.out.print("Tempo da operacao: ");
        System.out.printf("%.9f\n", sec);
	    
        return response;
    }
    
    //Operacao 3
     public String retornaInfo(String isbn) {
    	//obtem tempo inicial
    	inicio = System.nanoTime();
   	
    	//processa requisicao
        String response = new String();
        for (int i = 0; i < ISBN.size(); i++) {
            String str = ISBN.get(i);
            if(str.compareTo(isbn) == 0){
                response = isbn + "\n";
                response += titulo.get(i) + "\n";
                response += descricao.get(i) + "\n";
                response += estoque.get(i) + "\n";
                response += autor.get(i) + "\n";
                response += editora.get(i) + "\n";
                response += ano.get(i);
            }
        }
		
        //obtem tempo final e imprime 
    	fim = System.nanoTime();
        sec = fim-inicio;
        sec /= 1000000000;
        System.out.print("Tempo da operacao: ");
        System.out.printf("%.9f\n", sec);
		
        return response;
    }

    //operacao 4
    public String retornaTudo() {
    	//obtem tempo inicial
    	inicio = System.nanoTime();
    	
        //processa requisicao
        String response = new String();
        for (int i = 0; i < ISBN.size(); i++) {
            response += ISBN.get(i) + "\n";
            response += titulo.get(i) + "\n";
            response += descricao.get(i) + "\n";
            response += estoque.get(i) + "\n";
            response += autor.get(i) + "\n";
            response += editora.get(i) + "\n";
            response += ano.get(i) + "\n";          
        }

        //obtem tempo final e imprime 
    	fim = System.nanoTime();
        sec = fim-inicio;
        sec /= 1000000000;
        System.out.print("Tempo da operacao: ");
        System.out.printf("%.9f\n", sec);

        return response;
		
    }

    //operacao 5
    public void alteraEstoque(String isbn, int valor) {
    	//obtem tempo inicial
    	inicio = System.nanoTime();

    	//processa requisicao
    	String response = new String();
        for (int i = 0; i < ISBN.size(); i++) {
            String str = ISBN.get(i);
            if(str.compareTo(isbn) == 0){
                estoque.set(i, valor);
            }
        }

        //obtem tempo final e imprime 
    	fim = System.nanoTime();
        sec = fim-inicio;
        sec /= 1000000000;
        System.out.print("Tempo da operacao: ");
        System.out.printf("%.9f\n", sec);
    }

    //operacao 6
    public int retornaEstoque(String isbn) {
        int estoqueVal = 0;
    	//obtem tempo inicial
    	inicio = System.nanoTime();

    	//processa requisicao
    	String response = new String();
        for (int i = 0; i < ISBN.size(); i++) {
            String str = ISBN.get(i);
            if(str.compareTo(isbn) == 0){
                estoqueVal = estoque.get(i);
            }
        }

        //obtem tempo final e imprime 
    	fim = System.nanoTime();
        sec = fim-inicio;
        sec /= 1000000000;
        System.out.print("Tempo da operacao: ");
        System.out.printf("%.9f\n", sec);
		
        return estoqueVal;
    }

	
    public static void main(String args[]) {
	
	try {
	    Server obj = new Server();
	    Biblio stub = (Biblio) UnicastRemoteObject.exportObject(obj, 0);

	    // Bind the remote object's stub in the registry
	    Registry registry = LocateRegistry.getRegistry();
	    registry.bind("Biblio", stub);
	    System.err.println("Server ready");
	} catch (Exception e) {
	    System.err.println("Server exception: " + e.toString());
	    e.printStackTrace();
	}
    }
}
\end{verbatim}


\section*{Anexo 2: Código do Cliente}

\begin{verbatim}
package example.biblio;

import example.biblio.Biblio;
import java.rmi.registry.LocateRegistry;
import java.rmi.registry.Registry;
import java.util.Scanner;
import java.util.InputMismatchException;

public class Client {

    private Client() {}

    public static void main(String[] args) {

	String host = (args.length < 1) ? null : args[0];
	try {
	    Registry registry = LocateRegistry.getRegistry(host);
	    Biblio stub = (Biblio) registry.lookup("Biblio");
            
	    //loop de leitura do teclado e processamento das requisicoes
	    while(true){
                System.out.println("1- listar todos os ISBN e seus respectivos titulos");
                System.out.println("2- dado o ISBN de um livro, retornar descricao");
                System.out.println("3- dado o ISBN de um livro, retornar todas as informacoes do livro");
                System.out.println("4- listar todas as informacoes de todos os livros");
                System.out.println("5- alterar numero de exemplares em estoque");
                System.out.println("6- dado o ISBN de um livro, retornar numero de exemplares em estoque");
                System.out.println("7- fechar cliente");

                Scanner sc = new Scanner(System.in);
                int opt = 0;
                try{
                    System.out.println("Digite o numero da operacao");
                    opt = sc.nextInt();
                } catch (InputMismatchException e) {
                    System.out.println("Digite o numero da operacao");
                }
	    	
	    	String response = new String(); //resposta do servidor
	    
	    	//contagem de tempo
	    	long inicio = 0, fim = 0;
	    	double sec;
	    	
	    	if(opt == 1){
                    //contagem de tempo e chamada
                    inicio = System.nanoTime();
                    response = stub.listaISBN();
                    fim = System.nanoTime();
	    	}
	    	if(opt == 2){
                    System.out.println("Digite o ISBN do livro");
                    String isbn = sc.next();
                    //contagem de tempo e chamada
                    inicio = System.nanoTime();
                    response = stub.retornaDescricao(isbn);
                    fim = System.nanoTime();
	    	}
                
	    	if(opt == 3){
                    System.out.println("Digite o ISBN do livro");
                    String isbn = sc.next();
                    //contagem de tempo e chamada
                    inicio = System.nanoTime();
                    response = stub.retornaInfo(isbn);
                    fim = System.nanoTime();
	    	}

                if(opt == 4){
                    //contagem de tempo e chamada
                    inicio = System.nanoTime();
                    response = stub.retornaTudo();
                    fim = System.nanoTime();
                }
                
                if(opt == 5){
                    System.out.println("Digite a senha");
                    String senha = sc.next(); // verifica se cliente tem autorizacao
                    if(senha.equals("senhalivraria")){
                        System.out.println("Digite o ISBN do livro");
                        String isbn = sc.next();
                        System.out.println("Informe o novo valor do estoque");
                        int valor = sc.nextInt();
                        //contagem de tempo e chamada
                        inicio = System.nanoTime();
                        stub.alteraEstoque(isbn, valor);
                        fim = System.nanoTime();
                    }
                    else {
                        System.out.println("Senha invalida.");
                    }
                }

                if(opt == 6){
                    System.out.println("Digite o ISBN do livro");
                    String isbn = sc.next();
                    //contagem de tempo e chamada
                    inicio = System.nanoTime();
                    response += stub.retornaEstoque(isbn);
                    fim = System.nanoTime();
                }
                
                if(opt == 7){
                    System.exit(0);
                }
	    	
	    	System.out.println(response);
	    	sec = fim-inicio;
	    	sec /= 1000000000;
	    	System.out.print("Tempo da operacao: ");
	    	System.out.printf("%.9f\n", sec);
	    }
            
	    
	} catch (Exception e) {
	    System.err.println("Client exception: " + e.toString());
	    e.printStackTrace();
	}
    }
}
\end{verbatim}

\section*{Anexo 3: Código da Interface}
\begin{verbatim}
package example.biblio;

import java.rmi.Remote;
import java.rmi.RemoteException;

public interface Biblio extends Remote {
    String listaISBN() throws RemoteException;
    String retornaDescricao(String isbn) throws RemoteException;
    String retornaInfo(String isbn) throws RemoteException;
    String retornaTudo() throws RemoteException;
    void alteraEstoque(String isbn, int valor) throws RemoteException;
    int retornaEstoque(String isbn) throws RemoteException;
}
\end{verbatim}
\end{document}
